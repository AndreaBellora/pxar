% ----------------------------------------------------------------------
\section{User tests}
\label{s:usertests}
% ----------------------------------------------------------------------
It is straightforward to add your own tests to \pxar: you need to
write the source code and provide the configuration for
\testparameters. 
\begin{itemize}
\item The source code for all user tests are is located in
  \usertests. It is recommended that the name of your class starts
  with {\tt PixTest}. Provide the class definition in a header file ,
  whose name (without the extension {\tt .hh}) corresponds to the
  class name. Provide the implementation in a source file with
  extension {\tt .cc}. Following these recommendations allows that the
  `glob' in {\tt usertests/CMakeLists.txt} file picks up all user test classes in
  \usertests, and there is no need to manually insert your filenames
  into {\tt usertests/CMakeLists.txt}. 
\item To instantiate your test class in the GUI, and to provide the
  initial configuration of the test parameters, provide a section for
  \testparameters. 
\end{itemize}
After inserting the header and implementation files into \usertests,
go to your build directory, re-run the \cmake command, and
compile. The \cmake step will (re)build the {\tt
  pxar/usertests/PixUserTestFactory.cc} file, which will make your
test class visible, for instance in the GUI.
