% ======================================================================
\section{Installing pxar}
\label{s:install}
% ======================================================================

% ----------------------------------------------------------------------
\subsection{Dependencies}
\label{ss:overview}

\pxar has relatively few dependencies on other software, but some
features do rely on other packages. To configure the \pxar build process, the CMake cross-platform, open-source build system is used. The libusb and FTDI libraries are needed to communicate over USB with a DTB.

\subsubsection{CMake}
In order to automatically generate configuration files for the build process of \pxar both compiler and platform independent, the CMake build system is used.

CMake is available for all major operating systems from \url{http://www.cmake.org/cmake/resources/software.html}. On most Linux distributions it can usually be installed via the built-in package manager (aptitude/apt-get/yum etc.) and on OSX using
packages provided by e.g. the MacPorts or Fink projects.

Make sure to have CMake version 2.8.12 or above.

\subsubsection{C++ compiler}
The compilation of the \pxar source code requires a C++ compiler and has been tested with GCC, Clang, and MSVC (Visual
Studio 2012 and later) on Linux, OS X and Windows.

\subsubsection{ROOT}
The graphical user interface of \pxar, as well as a few command-line utilities,
use the Root package for histogramming. It can be downloaded from \url{http://root.cern.ch} or installed via
your favorite package manager. Make sure Root's \texttt{bin} subdirectory is in your path, so that the \texttt{root-config} utility can be run. This can be done by sourcing the \texttt{thisroot.sh} (or \texttt{thisroot.ch} for csh-like shells)
script in the \texttt{bin} directory of the Root installation:
\begin{verbatim}
source /path/to/root/bin/thisroot.sh
\end{verbatim}
Currently only ROOT 5.34/xx is supported. It is strongly recommended to build ROOT from the sources with the very same compiler which will be used to compile \pxar.

\subsubsection{libusb 1.0}
In order to communicate with the \gls{DTB}, the libusb library is needed. Please make sure that libusb is properly installed.

On Mac OS X, this can be installed using Fink or MacPorts. If using MacPorts you may also need to install the \texttt{libusb-compat} package. On Linux it may already be installed, otherwise you should use the built-in package manager to install it.
Make sure to get the development version, which may be named \texttt{libusb-1.0-dev} or \texttt{libusb-1.0-devel} instead of simply \texttt{libusb-1.0}.

\subsubsection{FDT2XX / FTDI library}
The DTB features a FTDI USB chip which needs the a special driver to communicate over USB. There are two options available. The FTDI library is a open source driver for the FTDI chips and is shipped with most Linux distributions and can usually installed via the package manager. The package is usually called \texttt{libftdi-dev} or similar. However, this driver has shown performance problems in the past and should only be used as fallback in case the proprietary driver does not work for some reason.

The proprietary driver privided by the FTDI company is called FTD2XX. Its binaries and header files can be fetched from \url{http://www.ftdichip.com/Drivers/D2XX.htm}. Make sure to pick the correct version for your operating system and system architecture. The dirver library and header files should be placed in the usual install locations of your system, e.g. for a Linux distribution usually \texttt{/usr/local/lib} and \texttt{/usr/local/include} are a good choice since those directories are not supervised by the package manager but the files are discovered by CMake. The files from the package are required:
\begin{verbatim}
ftd2xx.h
WinTypes.h
libftd2xx.so
\end{verbatim}
It might be necessary to crea a symlink pointing to libftd2xx.so (without the additional version numbers).

\subsubsection{Python, Cython, Numpy}
In case you want to use the Python interface to the \pxar core library to use the Pythin Command Line Interface or write short and simple scripts to program the detector and perform tests, you need Python 2.7, the python libraries, as well as Cython (version 0.19 or later) and the Python-Numpy package. All should be available for Linux (via the package manager), OS X and Windows.


\subsection{Downloading the source code}
The \pxar source code is hosted on github. The recommended way to obtain the software is with git, since this will allow you to easily update to newer versions. The latest version can be checked out with the following command:
\begin{verbatim}
git clone https://github.com/psi46/pxar.git pxar
\end{verbatim}

This will create the directory \texttt{pxar}, and download the latest development version into it. If you already have a copy installed, and want to update it to the latest version, you do not need to clone the repository again, just change to the \texttt{pxar} directory use the command:
\begin{verbatim}
git pull
\end{verbatim}
to update your local copy with all changes commited to the central repository.

Alternatively you can also download a zip file from
\url{https://github.com/psi46/pxar/archive/master.zip}.

For production environments (e.g. test stands and probe stations) we strongly recommend to use the latest release version. Use the command \texttt{git tag} in the repository to find the newest version and type e.g. 
\begin{verbatim}
git checkout tags/v1.4.0
\end{verbatim}
to change to version 1.4.0.

\subsection{Configuring via CMake}
CMake supports out-of-source configurations and builds -- just create and enter the './build' directory and run CMake, i.e.
\begin{verbatim}
mkdir build && cd build
cmake ..
\end{verbatim}

CMake automatically searches for all required packages and verifies that all dependencies are met using the \texttt{CMakeLists.txt} script in the main folder. By default, only the central shared library, the main executables including the graphical user interface (GUI) are configured for compilation. You can modify this default behavior by passing the \texttt{BUILD\_[name]} option to
CMake where \texttt{[name]} refers to an optional component, e.g.
\begin{verbatim}
cmake -DBUILD_pxarui=OFF -DBUILD_python=ON ..
\end{verbatim}
to disable the GUI but enable additionally the Cython interface to the core library.

The corresponding settings are cached, so that they will be again used
next time CMake is run.

Some of the optional packages include:
\begin{description}

\item{pxarui:}
The main executable and GUI. This requires the ROOT libraries and header files to be installed.

\item{tools:}
Additional tools ofr \pxar are build. Their source code resides in the \texttt{./tools} folder of the repository. This includes [FIXME]

\item{HV:}
[FIXME]

\item{dummydtb:}
[FIXME]

\item{python:}
[FIXME]

\end{description}

To install the binaries and the library outside the source tree, you
need to set the \texttt{INSTALL\_PREFIX} option, e.g.
\begin{verbatim}
cmake -D INSTALL_PREFIX=/usr/local ..
\end{verbatim}
to install the executables into the \texttt{bin} and the library into \texttt{lib} subdirectories of \texttt{/usr/local}.

If you ever need to, you can safely remove all files from the build folder as it only contains automatically generated files. Just run
\begin{verbatim}
cd build
rm -rf *
\end{verbatim}
to start from scratch.


% ----------------------------------------------------------------------
\subsection{Setup}
\label{ss:setup}
The \pxar code is hosted at \url{https://github.com/psi46/pxar}. Get it with 
{\tt
\begin{verbatim}
git clone git@github.com:psi46/pxar 
cd pxar
mkdir build
\end{verbatim}
}
If you run into problems here, you likely should setup git correctly
(see section~\ref{ss:installation}). 

For the time being, it is recommended that you get the HEAD of the
master branch. 


% ----------------------------------------------------------------------
\subsection{Compilation}
\label{ss:compilation}
Assuming that you start from the {\tt pxar} directory, do
{\tt
\begin{verbatim}
cd build
cmake -DBUILD_pxarui=ON ..
make [-j4] install [VERBOSE=1]
\begin{verbatim}
}
This will install by default the library into pxar/lib and the
executables into pxar/bin 
(both directories are subfolders of your local pxar folder).

Other cmake build options include
{\tt
\begin{verbatim}
cmake -DBUILD_pxarui=OFF ..
cmake -DBUILD_dummydtb=ON ..
\end{verbatim}
}

% ----------------------------------------------------------------------
\subsection{DTB firmware}
\label{ss:firmware}


To properly operated the DTB, you also need to have the matching
firmware loaded onto that device. The firmware can be obtained from 
\url{https://github.com/psi46/pixel-dtb-firmware/tree/master/FLASH} and the loaded
onto the DTB (its FPGA) with the following statements (assuming that
you are in directory {\tt pxar}): 
{\tt
\begin{verbatim}
cd ..
git clone git@github.com:psi46/pixel-dtb-firmware.git
cd pxar/main
../bin/pXar -f ../../pixel-dtb-firmware/FLASH/FLASHFILE
\end{verbatim}
} 
where {\tt FLASHFILE} should be replaced with the name of the most
recent version. The download will take a while. Follow the
instructions at the end: Wait until all 4 LEDs turn off and
power-cycle the DTB.

% ----------------------------------------------------------------------
\subsection{Program execution}
\label{ss:run}
The simplest way to run something of the \pxar framework is to use the
standalone test program: 
{\tt
\begin{verbatim}
../bin/testpxar
\end{verbatim}
}

To run the GUI, do 
{\tt
\begin{verbatim}
../bin/pXar -d ../data/defaultParametersRocPSI46digV2 -g [-v WARNING|QUIET|...]
../bin/pXar -d ../data/defaultParametersModulePSI46digV2 -g [-v WARNING|QUIET|...]
\end{verbatim}
}

and without the GUI, using a 'standard' ROOT C++ macro
{\tt
\begin{verbatim}
../bin/pXar -d ../data/defaultParametersRocPSI46digV2 \
  -c '../scripts/singleTest.C("DacScan", "DacScan.root")'
\end{verbatim}
}
